В соответствии с выданным вариантом на основе предложенного текстового отрывка из литературного произведения создать объектную модель реального или воображаемого мира, описываемого данным текстом. Должны быть выделены основные персонажи и предметы со свойственным им состоянием и поведением. На основе модели написать программу на языке Java. 

Текст, выводящийся в результате выполнения программы не обязан дословно повторять текст, полученный в исходном задании. Также не обязательно реализовывать грамматическое согласование форм и падежей слов выводимого текста. 

Стоит отметить, что цель разработки объектной модели состоит не в выводе текста, а в эмуляции объектов предметной области, а именно их состояния (поля) и поведения (методы). Методы в разработанных классах должны изменять состояние объектов, а выводимый текст должен являться побочным эффектом, отражающим эти изменения.

\textit{Он положил на стол перед рассевшимися вокруг коротышками лунный камень и принялся рассказывать о том, что в природе встречаются вещества, которые приобретают способность светиться в темноте, после того как подвергнутся действию лучей света. Такое свечение называется люминесценцией. Некоторые вещества приобретают способность испускать видимые лучи света даже под влиянием невидимых ультрафиолетовых, инфракрасных или космических лучей.}