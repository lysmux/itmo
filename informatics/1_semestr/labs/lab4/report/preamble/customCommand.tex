% НОВЫЕ КОМАНДЫ
\newcommand{\deriv}[2]{\frac{\partial #1}{\partial #2}}
\newcommand{\n}{\par}
\newcommand{\percent}{\mathbin{\%}}

% Заменяем Рис. на Рисунок
\addto\captionsrussian{\renewcommand{\figurename}{Рисунок}}

% Изменение формата подписей
% Стиль номера таблицы/рисунка #1-Таблица/Рис. #2-номер
\DeclareCaptionLabelFormat{custom}
{%
	#1 #2
}
% Стиль разделителя номера таблици/рисунка и названия таблицы/рисунка
\DeclareCaptionLabelSeparator{custom}{$-$}
% Стиль формата #1-номер таблицы/рисунка #2-разделитель #3-название
\DeclareCaptionFormat{custom}
{%
	#1 #2 #3
}

\captionsetup
{
	format=custom,%
	labelsep=custom,
	labelformat=custom
}

% ПЕРЕГРУЗКА УЖЕ СУЩЕСТВУЮЩИХ КОМАНД
\renewcommand{\epsilon}{\varepsilon} % Заменить знак эпсилон
\renewcommand{\phi}{\varphi}
\renewcommand{\kappa}{\varkappa}
\renewcommand{\lambda}{\uplambda}
