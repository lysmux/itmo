\input{preamble/preamble.tex}

\begin{document}
	\newcommand{\Faculty}{Факультет программной инженерии и компьютерной техники}

\newcommand{\TeacherPosition}{}
\newcommand{\TeacherName}{Белокон Юлия Алексеевна}

\newcommand{\LabSubject}{Информатика}
\newcommand{\LabNumber}{№4}
\newcommand{\LabName}{Исследование протоколов, форматов обмена информацией и языков разметки документов}
\newcommand{\Variant}{13}
 
\newcommand{\StudentGroup}{Р3115}
\newcommand{\StudentName}{Разыграев Кирилл Сергеевич}


%%% Title page
\thispagestyle{empty}
\setlength{\parindent}{0cm} % First line margin

\begin{figure}[h]
	\centering
	\includegraphics{logo}
\end{figure}
\vspace{-\baselineskip}


\begin{center}
	Федеральное государственное автономное образовательное \\
	учреждение высшего образования\\
	«Национальный исследовательский университет ИТМО»
\end{center}\par

\begin{center}
	\vspace{12pt}
	\Faculty
\end{center}\par

\vspace{\fill}
\begin{center}
	Отчёт по лабораторной работе \LabNumber \\
	По дисципление: \LabSubject \\
	Тема: <<\LabName>> \\
	Вариант (по ИСУ) \Variant
\end{center}\par

\vspace{\fill}
\vbox{
	\hfill
	\vbox{
		\hbox{\textbf{Выполнил:} \StudentName}
		\hbox{\textbf{Группа:} \StudentGroup \\}
		\hbox{\textbf{Преподаватель:} \TeacherPosition \TeacherName}
	}
} 


\vspace{\fill}
\begin{center}
	Санкт-Петербург, \the\year{}
\end{center}\par

\newpage

\setlength{\parindent}{1.25cm} % Restore first line margin

	
	\tableofcontents
	\newpage
	
	\section{Задание}
	\begin{enumerate}
	\item Создать приведенное в варианте дерево каталогов и файлов с содержимым. В качестве корня дерева использовать каталог lab0 своего домашнего каталога. Для создания и навигации по дереву использовать команды: mkdir, echo, cat, touch, ls, pwd, cd, more, cp, rm, rmdir, mv.
	\begin{figure}[h]
		\centering
		\includegraphics[scale=0.5]{task}
	\end{figure}
	
	\item Установить согласно заданию права на файлы и каталоги при помощи команды chmod, используя различные способы указания прав.
	\begin{itemize}
		\item camerupt4: владелец должен не иметь никаких прав; группа-владелец должна не иметь никаких прав; остальные пользователи должны читать файл
		 \item conkeldurr6: владелец должен читать и записывать файл; группа-владелец должна записывать файл; остальные пользователи должны записывать файл
		 \item crawdaunt7: -wxrwxr-x
		 \item slaking: владелец должен не иметь никаких прав; группа-владелец должна читать и записывать файл; остальные пользователи должны записывать файл
		 \item skiploom: владелец должен не иметь никаких прав; группа-владелец должна читать файл; остальные пользователи должны читать и записывать файл
		 \item duosion: права 555
		 \item golem: права 337
		 \item nincada: права 555
		 \item wormadam: ------rw-
		 \item parasect2: -wx-wxr-x
		 \item flaaffy: rwxr-x-w-
		 \item seel: владелец должен читать и записывать файл; группа-владелец должна читать файл; остальные пользователи должны не иметь никаких прав
		 \item vaporeon: владелец должен читать файл; группа-владелец должна читать файл; остальные пользователи должны не иметь никаких прав
		 \item darumaka: владелец должен читать, записывать директорию и переходить в нее; группа-владелец должна читать директорию и переходить в нее; остальные пользователи должны записывать директорию
		 \item pikachu: владелец должен не иметь никаких прав; группа-владелец должна читать файл; остальные пользователи должны читать и записывать файл
		 \item accelgor: права 044
		 \item samurott1: r--------
		 \item zorua3: r-x--x-wx
		 \item garbodor: владелец должен читать и записывать файл; группа-владелец должна не иметь никаких прав; остальные пользователи должны не иметь никаких прав
		 \item misdreavus: владелец должен читать директорию и переходить в нее; группа-владелец должна только переходить в директорию; остальные пользователи должны записывать директорию и переходить в нее
		 \item leafeon: r-----r--
		 \item cherrim: владелец должен читать директорию и переходить в нее; группа-владелец должна читать, записывать директорию и переходить в нее; остальные пользователи должны читать, записывать директорию и переходить в нее
		 \item golduck: права 640
	\end{itemize}
	
	\item Скопировать часть дерева и создать ссылки внутри дерева согласно заданию при помощи команд cp и ln, а также комманды cat и перенаправления ввода-вывода.
	\begin{itemize}
		\item скопировать рекурсивно директорию parasect2 в директорию lab0/crawdaunt7/nincada
		\item скопировать файл camerupt4 в директорию lab0/crawdaunt7/duosion
		\item cоздать жесткую ссылку для файла conkeldurr6 с именем lab0/crawdaunt7/slakingconkeldurr
		\item создать символическую ссылку c именем Copy\_5 на директорию zorua3 в каталоге lab0
		\item объеденить содержимое файлов lab0/zorua3/golduck, lab0/parasect2/accelgor, в новый файл lab0/camerupt4\_76
		\item скопировать содержимое файла conkeldurr6 в новый файл lab0/parasect2/accelgorconkeldurr
		\item cоздать символическую ссылку для файла camerupt4 с именем lab0/crawdaunt7/slakingcamerupt
	\end{itemize}
	
	\item Используя команды cat, wc, ls, head, tail, echo, sort, grep выполнить в соответствии с вариантом задания поиск и фильтрацию файлов, каталогов и содержащихся в них данных.
	\begin{itemize}
		\item Рекурсивно подсчитать количество символов содержимого файлов из директории lab0, имя которых начинается на 'c', результат записать в файл в директории /tmp, подавить вывод ошибок доступа
		\item Вывести четыре первых элемента рекурсивного списка имен и атрибутов файлов в директории lab0, содержащих строку "da", список отсортировать по убыванию даты модификации файла, подавить вывод ошибок доступа
		\item Рекурсивно вывести содержимое файлов из директории lab0, имя которых заканчивается на 'r', строки отсортировать по имени a->z, ошибки доступа перенаправить в файл в директории /tmp
		\item Вывести содержимое файлов: wormadam, seel, vaporeon, pikachu, accelgor, garbodor, leafeon, исключить строки, заканчивающиеся на 'n', регистр символов игнорировать, добавить вывод ошибок доступа в стандартный поток вывода
		\item Подсчитать количество символов содержимого файлов в директории crawdaunt7, результат записать в файл в директории /tmp, добавить вывод ошибок доступа в стандартный поток вывода
		\item Вывести два первых элемента рекурсивного списка имен и атрибутов файлов в директории lab0, содержащих строку "ldu", список отсортировать по возрастанию даты изменения записи о файле, ошибки доступа перенаправить в файл в директории /tmp
	\end{itemize}
	
	\item Выполнить удаление файлов и каталогов при помощи команд rm и rmdir согласно варианту задания.
	\begin{itemize}
		\item Удалить файл conkeldurr6
		\item Удалить файл lab0/crawdaunt7/wormadam
		\item удалить символические ссылки lab0/crawdaunt7/slakingcameru*
		\item удалить жесткие ссылки lab0/crawdaunt7/slakingconkeldu*
		\item Удалить директорию zorua3
		\item Удалить директорию lab0/parasect2/flaaffy
	\end{itemize}
\end{enumerate}
	
	\section{Исходный файл}
	\inputminted{xml}{../schedule.xml}
	
	\section{Основные этапы вычисления}
	\subsection{Обязательное задание}
	\subsubsection{Исходный код}
\inputminted[breaklines]{python}{../task_1.py}

\subsubsection{Результат}
\inputminted[breaklines]{js}{../schedule_1.json}
	
	\subsection{Дополнительное задание \textnumero 1}
	\subsection*{Задание}

\begin{enumerate}
\item Реализуйте программный продукт на языке Python, используя регулярные выражения по варианту, представленному в таблице.
\item Для своей программы придумайте минимум 5 тестов. Каждый тест является отдельной сущностью, передаваемой регулярному выражению для обработки. Для каждого теста необходимо самостоятельно (без использования регулярных выражений) найти правильный ответ. После чего сравнить ответ, выданный программой, и полученный самостоятельно. Все 5 тестов необходимо показать при защите. Пример тестов приведён в таблице.
\item Можно использовать циклы и условия, но основной частью решения должны быть регулярные выражения.
\end{enumerate}

\begin{figure}[h]
	\centering
	\includegraphics[scale=0.7]{task_2}
\end{figure}

\subsection*{Тестовые файлы}
\subsubsection*{Тест 1}
\textit{Кривошеее существо гуляет по парку}

\subsubsection*{Тест 2}
\textit{Кривоногое создание с интересом рассматривало яркое солнце над горизонтом}

\subsubsection*{Тест 3}
\textit{Синее небо раскрылось перед маленьким озером, окруженным камнями. Двуххвостое животное лениво отдыхало под тихим ветром на берегу моря.}

\subsubsection*{Тест 4}
\textit{Безухое создание медленно пробиралось через тёплую траву на поляне.}

\subsubsection*{Тест 5}
\textit{Пятиногое создание медленно двигалось через зелёную поляну у леса.}

\subsection*{Листинг}
\begin{minted}[breaklines]{python}
import re
from pathlib import Path

TEXTS_PATH = Path(__file__).parent / "texts" / "task_2"
PATTERN = re.compile(
    r"\b\w*[аеёиоуыэюя]{2}\w*\b(?!\s+(?:[аеёиоуыэюя]*[бвгджзйклмнпрстфхцчшщ]){4,})",
    flags=re.MULTILINE
)
CORRECT_ANSWERS = (
    {"гуляет"},
    {"создание"},
    {"Синее", "животное"},
    {"тёплую"},
    {"зелёную"}
)


def find_words(text: str) -> set[str]:
    return set(PATTERN.findall(text))


def main() -> None:
    for i in range(1, 6):
    text = (TEXTS_PATH / f"text_{i}").read_text(encoding="utf-8")
    words = find_words(text)

    if words != CORRECT_ANSWERS[i - 1]:
        print(
            f"Ошибка в тесте {i}. "
            f"Ожидалось - {CORRECT_ANSWERS[i - 1]}, "
            f"Результат - {words}"
        )
    else:
        print(f"Текст {i}: слова - {words}")
    print("=" * 100)


if __name__ == '__main__':
    main()
\end{minted}


	
	\subsection{Дополнительное задание \textnumero 2}
	\subsubsection{Исходный код}
\inputminted[breaklines]{python}{../task_3.py}

\subsubsection{Результат}
\inputminted[breaklines]{js}{../schedule_3.json}
Результаты работы ручного парсинга и парсинга с использованием регулярных выражений совпадают. Применение регулярных выражений позволило сократить объем кода, но может затруднить его понимание для тех, кто не знаком с их синтаксисом.
	
	\subsection{Дополнительное задание \textnumero 3}
	\subsubsection{Исходный код}
\inputminted[breaklines]{python}{../task_4.py}

\subsubsection{Результат}
\inputminted[breaklines]{js}{../schedule_4.json}

\subsubsection{Сравнение}
Результаты парсинга с использованием формальных грамматик совпали с результатами, полученными ранее. Применение формальных грамматик привело к увеличению объема кода, однако это дало существенные преимущества. Теперь появилась возможность детально управлять процессом парсинга для каждого отдельного компонента, что повышает гибкость и упрощает внесение изменений или расширение функциональности в будущем.
	
	\subsection{Дополнительное задание \textnumero 4}
	\subsubsection{Исходный код}
\inputminted[breaklines]{python}{../task_5.py}

\subsubsection{Результат}
\begin{itemize}
	\item Без использования библиотек: 0.006938100035768002
	\item С использование библиотек: 0.012030399986542761
	\item С использованием регулярных выражений: 0.014387500006705523
	\item С использованием формальных грамматик: 0.03452159999869764
\end{itemize}

\subsubsection{Сравнение}
Ручной парсинг оказался самым быстрым (\textit{0.0069 с}), но менее универсальным. Использование библиотек (\textit{0.012 с}) немного увеличило время, обеспечив удобство и надёжность. Регулярные выражения (\textit{0.014 с}) добавили гибкость, но усложнили понимание. Формальные грамматики оказались самыми медленными (\textit{0.034 с}), однако предоставили максимальный контроль и расширяемость.
	
	\subsection{Дополнительное задание \textnumero 5}
	\subsubsection{Исходный код}
\inputminted[breaklines]{python}{../task_6.py}

\subsubsection{Результат}
\inputminted[breaklines]{text}{../schedule.csv}

\subsection{Сравнение}
Изначальный JSON-файл был преобразован в CSV с использованием разделителя ;. Формат CSV удобен для представления данных в табличной форме, что облегчает их чтение, анализ и обработку в большинстве стандартных инструментов для работы с таблицами
	
	\section{Вывод}
	В ходе лабораторной работы был приобретён практический опыт преобразования форматов данных с использованием как готовых библиотек, так и самописных алгоритмов. Это позволило сравнить их производительность и удобство  в различных сценариях.
	
	\section{Список использованных источников}
	\begin{enumerate}
		\item 	Балакшин П.В., Соснин В.В., Калинин И.В., Малышева Т.А., Раков С.В., Рущенко Н.Г., Дергачев А.М. Информатика: лабораторные работы и тесты: Учебно-методическое пособие / Рецензент: Поляков В.И. - Санкт-Петербург: Университет ИТМО, 2019. - 56 с
		
		\item Грошев А.С. Г89 Информатика: Учебник для вузов / А.С. Грошев. – Архангельск, Арханг. гос. техн. ун-т, 2010. -470с.
	\end{enumerate}
	
\end{document}