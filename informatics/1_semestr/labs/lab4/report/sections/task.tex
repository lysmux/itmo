\begin{itemize}
	\item  \textbf{Обязательное задание}. написать программу на языке Python 3.x или любом другом, которая бы осуществляла парсинг и конвертацию исходного файла в новый путём простой замены метасимволов исходного формата на метасимволы результирующего формата
	
	Нельзя использовать готовые библиотеки, в том числе регулярные выражения в Python и библиотеки для загрузки XML-файлов
	
	\item \textbf{Дополнительное задание \textnumero 1}
	\begin{itemize}
		\item Найти готовые библиотеки, осуществляющие аналогичный парсинг и конвертацию файлов
		
		\item Переписать исходный код, применив найденные библиотеки. Регулярные выражения также нельзя использовать
		
		\item Сравнить полученные результаты и объяснить их сходство/различие. Объяснение должно быть отражено в отчёте
	\end{itemize}
	
	\item \textbf{Дополнительное задание \textnumero 2}
	\begin{itemize}
		\item Переписать исходный код, добавив в него использование регулярных выражений

		\item Сравнить полученные результаты и объяснить их сходство/различие. Объяснение должно быть отражено в отчёте
	\end{itemize}
	
	\item \textbf{Дополнительное задание \textnumero 3}
	\begin{itemize}
		\item Переписать исходный код таким образом, чтобы для решения задачи использовались формальные грамматики. То есть ваш код должен уметь осуществлять парсинг и конвертацию любых данных, представленных в исходном формате, в данные, представленные в результирующем формате: как с готовыми библиотеками из дополнительного задания \textnumero 1
		
		\item Проверку осуществить как минимум для расписания с двумя учебными днями по два занятия в каждом
		
		\item Сравнить полученные результаты и объяснить их сходство/различие. Объяснение должно быть отражено в отчёте
	\end{itemize}
	
	\item \textbf{Дополнительное задание \textnumero 4}
	\begin{itemize}
		\item Используя свою исходную программу из обязательного задания и программы из дополнительных заданий, сравнить стократное время выполнения парсинга + конвертации в цикле
		
		\item Сравнить полученные результаты и объяснить их сходство/различие. Объяснение должно быть отражено в отчёте
	\end{itemize}
	
	\item \textbf{Дополнительное задание \textnumero 5}
	\begin{itemize}
		\item Переписать исходную программу, чтобы она осуществляла парсинг и конвертацию  файла в любой другой формат (кроме JSON, YAML, XML, HTML): PROTOBUF, TSV, CSV, WML и т.п.
		
		\item Сравнить полученные результаты и объяснить их сходство/различие. Объяснение должно быть отражено в отчёте
	\end{itemize}
\end{itemize}
